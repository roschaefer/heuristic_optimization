%%%%%%%%%%%%%%%%%%%%%%%%%%%%%%%%%%%%%%%%%
% Programming/Coding Assignment
% LaTeX Template
%
% This template has been downloaded from:
% http://www.latextemplates.com
%
% Original author:
% Ted Pavlic (http://www.tedpavlic.com)
%
% Note:
% The \lipsum[#] commands throughout this template generate dummy text
% to fill the template out. These commands should all be removed when
% writing assignment content.
%
% This template uses a Perl script as an example snippet of code, most other
% languages are also usable. Configure them in the "CODE INCLUSION
% CONFIGURATION" section.
%
%%%%%%%%%%%%%%%%%%%%%%%%%%%%%%%%%%%%%%%%%

%----------------------------------------------------------------------------------------
%	PACKAGES AND OTHER DOCUMENT CONFIGURATIONS
%----------------------------------------------------------------------------------------

\documentclass{article}

\usepackage{fancyhdr} % Required for custom headers
\usepackage{lastpage} % Required to determine the last page for the footer
\usepackage{extramarks} % Required for headers and footers
\usepackage[usenames,dvipsnames]{color} % Required for custom colors
\usepackage{graphicx} % Required to insert images
\usepackage{listings} % Required for insertion of code
\usepackage{courier} % Required for the courier font
\usepackage{lipsum} % Used for inserting dummy 'Lorem ipsum' text into the template
\usepackage{hyperref}
\usepackage{amssymb}
\usepackage{amsmath}
\usepackage[utf8]{inputenc}

% Margins
\topmargin=-0.45in
\evensidemargin=0in
\oddsidemargin=0in
\textwidth=6.5in
\textheight=9.0in
\headsep=0.25in

\linespread{1.1} % Line spacing

% Set up the header and footer
\pagestyle{fancy}
\lhead{\hmwkAuthorName} % Top left header
\rhead{\hmwkClass\ (\hmwkClassInstructor): \hmwkTitle} % Top right head
\lfoot{\lastxmark} % Bottom left footer
\cfoot{} % Bottom center footer
\rfoot{Page\ \thepage\ of\ \protect\pageref{LastPage}} % Bottom right footer
\renewcommand\headrulewidth{0.4pt} % Size of the header rule
\renewcommand\footrulewidth{0.4pt} % Size of the footer rule

\setlength\parindent{0pt} % Removes all indentation from paragraphs

%----------------------------------------------------------------------------------------
%	CODE INCLUSION CONFIGURATION
%----------------------------------------------------------------------------------------

\definecolor{MyDarkGreen}{rgb}{0.0,0.4,0.0} % This is the color used for comments
\lstloadlanguages{Perl} % Load Perl syntax for listings, for a list of other languages supported see: ftp://ftp.tex.ac.uk/tex-archive/macros/latex/contrib/listings/listings.pdf
\lstset{language=Perl, % Use Perl in this example
        frame=single, % Single frame around code
        basicstyle=\small\ttfamily, % Use small true type font
        keywordstyle=[1]\color{Blue}\bf, % Perl functions bold and blue
        keywordstyle=[2]\color{Purple}, % Perl function arguments purple
        keywordstyle=[3]\color{Blue}\underbar, % Custom functions underlined and blue
        identifierstyle=, % Nothing special about identifiers
        commentstyle=\usefont{T1}{pcr}{m}{sl}\color{MyDarkGreen}\small, % Comments small dark green courier font
        stringstyle=\color{Purple}, % Strings are purple
        showstringspaces=false, % Don't put marks in string spaces
        tabsize=5, % 5 spaces per tab
        %
        % Put standard Perl functions not included in the default language here
        morekeywords={rand},
        %
        % Put Perl function parameters here
        morekeywords=[2]{on, off, interp},
        %
        % Put user defined functions here
        morekeywords=[3]{test},
       	%
        morecomment=[l][\color{Blue}]{...}, % Line continuation (...) like blue comment
        numbers=left, % Line numbers on left
        firstnumber=1, % Line numbers start with line 1
        numberstyle=\tiny\color{Blue}, % Line numbers are blue and small
        stepnumber=5 % Line numbers go in steps of 5
}

% Creates a new command to include a perl script, the first parameter is the filename of the script (without .pl), the second parameter is the caption
\newcommand{\results}[2]{
\begin{itemize}
\item[]\lstinputlisting[caption=#2,label=#1]{#1}
\end{itemize}
}

%----------------------------------------------------------------------------------------
%	DOCUMENT STRUCTURE COMMANDS
%	Skip this unless you know what you're doing
%----------------------------------------------------------------------------------------

% Header and footer for when a page split occurs within a problem environment
\newcommand{\enterProblemHeader}[1]{
\nobreak\extramarks{#1}{#1 continued on next page\ldots}\nobreak
\nobreak\extramarks{#1 (continued)}{#1 continued on next page\ldots}\nobreak
}

% Header and footer for when a page split occurs between problem environments
\newcommand{\exitProblemHeader}[1]{
\nobreak\extramarks{#1 (continued)}{#1 continued on next page\ldots}\nobreak
\nobreak\extramarks{#1}{}\nobreak
}

\setcounter{secnumdepth}{0} % Removes default section numbers
\newcounter{homeworkProblemCounter} % Creates a counter to keep track of the number of problems

\newcommand{\homeworkProblemName}{}
\newenvironment{homeworkProblem}[1][Problem \arabic{homeworkProblemCounter}]{ % Makes a new environment called homeworkProblem which takes 1 argument (custom name) but the default is "Problem #"
\stepcounter{homeworkProblemCounter} % Increase counter for number of problems
\renewcommand{\homeworkProblemName}{#1} % Assign \homeworkProblemName the name of the problem
\section{\homeworkProblemName} % Make a section in the document with the custom problem count
\enterProblemHeader{\homeworkProblemName} % Header and footer within the environment
}{
\exitProblemHeader{\homeworkProblemName} % Header and footer after the environment
}

\newcommand{\problemAnswer}[1]{ % Defines the problem answer command with the content as the only argument
\noindent\framebox[\columnwidth][c]{\begin{minipage}{0.98\columnwidth}#1\end{minipage}} % Makes the box around the problem answer and puts the content inside
}

\newcommand{\homeworkSectionName}{}
\newenvironment{homeworkSection}[1]{ % New environment for sections within homework problems, takes 1 argument - the name of the section
\renewcommand{\homeworkSectionName}{#1} % Assign \homeworkSectionName to the name of the section from the environment argument
\subsection{\homeworkSectionName} % Make a subsection with the custom name of the subsection
\enterProblemHeader{\homeworkProblemName\ [\homeworkSectionName]} % Header and footer within the environment
}{
\enterProblemHeader{\homeworkProblemName} % Header and footer after the environment
}

%----------------------------------------------------------------------------------------
%	NAME AND CLASS SECTION
%----------------------------------------------------------------------------------------

\newcommand{\hmwkTitle}{Project\ \#4} % Assignment title
\newcommand{\hmwkDueDate}{Monday,\ July\ 13th,\ 2015} % Due date
\newcommand{\hmwkClass}{Heuristic Optimization} % Course/class
\newcommand{\hmwkClassInstructor}{Prof. Tobias Friedrich} % Teacher/lecturer
\newcommand{\hmwkAuthorName}{Willi M\"uller, Robert Sch\"afer} % Your name

%----------------------------------------------------------------------------------------
%	TITLE PAGE
%----------------------------------------------------------------------------------------

\title{
\vspace{2in}
\textmd{\textbf{\hmwkClass:\ \hmwkTitle}}\\
\normalsize\vspace{0.1in}\small{Due\ on\ \hmwkDueDate}\\
\vspace{0.1in}\large{\textit{\hmwkClassInstructor}}
\vspace{3in}
}

\author{\textbf{\hmwkAuthorName}}
\date{13.07.2015} % Insert date here if you want it to appear below your name

%----------------------------------------------------------------------------------------

\begin{document}

\maketitle


\newpage

%----------------------------------------------------------------------------------------
%	PROBLEM 1
%----------------------------------------------------------------------------------------

% To have just one problem per page, simply put a \clearpage after each problem
\section{Sudoku}

\subsection{Approach: Solving with Answer Set Programming}
We modeled the sudoku problem with a solver for Answer Set Programming, a declarative programming model complex combinatorial problems in NP, such as satisfiability.
We used the implementation of the Potassco Suite, provided by the University of Potsdam \url{http://potassco.sourceforge.net/}.

The usual programming model is to generate the solution space and filter invalid solutions with constraints.
The relevant part of our code is listed below.

\begin{verbatim}
% first row
initial(2,2,10).
initial(3,3,15).
%... and so on

% a valid solution has to include the initial values
sudoku(X,Y,N) :- initial(X,Y,N).

% range constraints for coordinates and numbers
1 { sudoku(1..16,Y,N) } 1 :- Y=1..16, N=1..16.
1 { sudoku(X,1..16,N) } 1 :- X=1..16, N=1..16.
1 { sudoku(X,Y,1..16) } 1 :- X=1..16, Y=1..16.

% N must not appear twice in a row, column and 4x4 square
:- sudoku(X1,Y1,N), sudoku(X2,Y2,N), (X1,Y1) != (X2,Y2),
(((X1-1)/4),((Y1-1)/4)) == (((X2-1)/4), ((Y2-1)/4)).

\end{verbatim}

The sudoku is modeled as a list of vectors with 3 elements respectively.
The first two elements correspond to the X and Y coordinate and the third element corresponds to the number N in the field.

The initial configuration \texttt{initial(X,Y,N)} ensures, that a \texttt{sudoku} term is generated with the given values in the assignment. As an example the first row is given in the code listing.
Then, the solver generates for every permutation of row (Y) and number (N) exactly one \texttt{sudoku(X,Y,N)} term, such that the chosen column(X) lies in $[1, 16]$.
Similarily, the solver permutates the column and number to find a row(Y) and permutates column and row to find a number(N), which also have to be in the range of $[1, 16]$.

The last constraint specifies that for two pairs of coordinates (X1, Y1) and (Y1, Y2), that (i) share the same number N, the coordinates have to be (ii) distinct if they lie in the same 4x4 square.

\subsection{Solution}
Some example solutions are provided in the file sudoku-solutions.txt.

\subsection{Conclusion}
This implementation was able to find valid sudoku solutions in less than 1 second. Therefore we conclude that the ASP solver is a well-suited technique to solve the given problem, due to its combinatorial characteristics.


%%%%%%%%%%%%%%%%%%%%%%%%%%%%%%%%%%%%%%%%
\clearpage
\section{Windfarm}


\subsection{Approach}
\subsubsection{Random Local Search}
We started to implement random local search and always continue with the new configuration if it is at least as good as the old configuration. In each iteration we modify a randomly chosen point by a random delta between $[-100,100] \in \mathbb{Z}$. In Figure~\ref{random} it is visible that this run finds its optimum fairly early, after around 100 iterations and then does not improve the cost of its found configurations.

\begin{figure}[h]
    \includegraphics[width=0.5\linewidth]{img/random-delta-100-2.png}
    \centering
    \caption{Cost over iterations for a random delta between $[-100,100]$.}
    \label{random}
\end{figure}


\subsubsection{Convolutional Search with Gauss Sampling}
We then modified our random local search: Instead of adding a random delta to a randomly chosen point we set each coordinate from a randomly chosen point (x, y) to a value sampled from a Gaussian distribution with $\mu$ being the current coordinate and standard deviation $\sigma$. If the cost is lower than the current optimum, we decrease the current standart deviation $\sigma$ by a factor $\alpha$.

From the Figures~\ref{s5000-min} and \ref{s1000-min} we cannot conclude a difference in final convergence between a standard deviation of $\sigma=5000$ or $\sigma=1000$. However, we tried with smaller values and observed that they do not get placed far enough from the initial configuration to find a better placement.

In Figure~\ref{s5000-no-min} we observe that setting a lower bound for the standard deviation sigma appears to have an effect: The cost ceases to converge too early. In fact, we observed during multiple runs that not decreasing $\sigma$ arbitrarily prolongs the phase of decreasing cost. The reason might most likely be that even during late iterations there is still a certain flexibility to place wind turbines, which is not the case when $\sigma$ can shrink arbitrarily small.

\begin{figure}[h]
    \includegraphics[width=0.5\linewidth]{img/gauss-s5000-min50.png}
    \centering
    \caption{Cost over iterations for $\sigma=5000$, $\sigma_{min}=50$, $\alpha=0.9$}
    \label{s5000-min}
\end{figure}

\begin{figure}[h]
    \includegraphics[width=0.5\linewidth]{img/gauss-s1000-min50.png}
    \centering
    \caption{Cost over iterations for $\sigma=1000$, $\sigma_{min}=50$, $\alpha=0.9$}
    \label{s1000-min}
\end{figure}

\begin{figure}[h]
    \includegraphics[width=0.5\linewidth]{img/gauss-s5000-min0.png}
    \centering
    \caption{Cost over iterations for $\sigma=5000$, $\sigma_{min}=0$, $\alpha=0.9$}
    \label{s5000-no-min}
\end{figure}


\subsection{Discussion}

The convolution approach using gaussian sampling moving only one turbine per iteration turned out to be most successful in terms of convergence because it continuously improves and produced the least cost. A parameter configuration of a fairly high $\sigma$ between 1000 and 5000 allows for variability in the turbine placement whereas a dampening factor of $\alpha=0.9$ for $\sigma$ after each cost improvement turned out to be the best parameter during our test runs.

The minimal solution found with these parameters is 0.0019941. However, a basic random local search also found a solution (cost = 0.00199964) better than the initial one (cost = 0.00199975).

\subsection{Conclusions}
Our first finding was that it is unfeasible to generate random samples and evaluate them with the provided cost function because a totally randomized sample too frequently violates the security constraint.
For this reason it is necessary to start with a user-defined initial configuration that fulfills both the bounding box and security constraints and then modify this configuration in each iteration.

We also conclude, that a convolutional aproach is beneficial over a pure random local search to achieve a continuous convergence and a better final solution. In order to continue decreasing the cost it is important to set a lower bound for the variability in the solution.

From our two implementations we thus conclude that that a fairly simple algorithm can be a good starting point before using more sophisticated implementations.


%----------------------------------------------------------------------------------------

\end{document}
